\hypertarget{index_intro_sec}{}\section{Introduction}\label{index_intro_sec}
This a custom R\-O\-O\-T library that provides a framework for physics analysis. It contains several functions and classes that aid in analyzing high energy physics data. To aid in the compilation of all the source code, a macro entitled {\ttfamily Compile\-Source.\-C} can be executed from the command line. All functions and classes are available in python from the {\ttfamily lib/\-H\-A\-L.\-py} module. Furthermore, everything lives in the '\hyperlink{namespace_h_a_l}{H\-A\-L}' namespace. \hyperlink{namespace_h_a_l}{H\-A\-L} stands for\par
 H -\/ H.\-E.\-P.\par
 A -\/ Analysis\par
 L -\/ Library\par
 Of course, this is also a thinly veiled reference to the villian in the classic \char`\"{}2001\-: A Space Odyssey.\char`\"{} However, this bit of software is far less pernicious.\hypertarget{index_overview_sec}{}\section{Overview}\label{index_overview_sec}
\hyperlink{namespace_h_a_l}{H\-A\-L}'s design goals are\-:\par
 Flexiblity -\/ Users have choice in how they use \hyperlink{namespace_h_a_l}{H\-A\-L}. A user can take full advantage of the builtin framework and wide selection of generic algorithms or just add a few helpful \hyperlink{namespace_h_a_l}{H\-A\-L} classes or functions to an existing analysis.\par
 Usability -\/ \hyperlink{namespace_h_a_l}{H\-A\-L} strives to make the user experience as physics-\/focused as possible. This means providing classes that do a lot of the heavy lifting, including\-: reading a T\-Tree and assigning branches, looping over events, making unique tuples from lists of particles, and more. The interfaces for these classes are designed for a smooth user experience.\par
 Extensibility -\/ The framework in \hyperlink{namespace_h_a_l}{H\-A\-L} is trivially extended with custom user code. This means that analysis-\/specific code can couple with generic algorithms with minimal effort.\par
 Performance -\/ \hyperlink{namespace_h_a_l}{H\-A\-L} leverages the performance advanages of C++. The framework is competitive with hand-\/coded analyses but with advanage of ample code reuse.\par
 